\section{Related Work}\label{sec:related}

We briefly review related work on home broadband networks.  We survey
work that has performed ``single shot'' measurements of home broadband
network performance and characteristics, Internet policy reports on
performance in various regions, and qualitative studies that lend
insight into how to better design home networks.

\fp {\bf Measuring home network performance.}  There has been
significant interest in measuring home and access networks, and many
previous studies have tried to measure home networks in various
ways. The studies use various methods for measuring home networks,
from running measurements remotely from servers in the Internet to
measure access link properties~\cite{Dischinger:imc2007} to running
tools on end-host
devices~\cite{DiCioccio:2012,DiCioccio2013,www-netalyzr,Kreibich2010,Sanchez2013,Chetty:2011:WMI}.
In contrast to these previous studies, we perform our measurements from
gateway routers, not end-host devices, which enables {\em continuous}
monitoring rather than one-shot measurements.  Such continuous
monitoring allows us to observe how usage patterns change over time,
both on short and long timescales, as well as to report on other
characteristics that require continuous monitoring, such as availability.
Our work builds on our own previous work~\cite{sundaresan2011}, which
uses a deployment of custom home access points that conduct a unique set
of performance measurements.  In contrast to our previous work, this study
broadly characterizes home network usage rather than focusing
on access link performance.

\fp {\bf National and regional Internet policy and measurement reports.}
Recently, there has been interest from national agencies to measure home
networks for Internet policy and regulation purposes.  The United States
Federal Communications Commission, United Kingdom's Ofcom and the
European Commission have all conducted large scale studies of access
networks in conjunction with
SamKnows~\cite{www-fcc-report,www-sk-eu,www-sk-uk}.  Benkler~\etal have
a report on broadband transitions and policies around the
world~\cite{RePEc:reg:wpaper:8}.  To date, these studies from regulatory
commissions have focused exclusively on performance of the access
network, rather than on properties of the home network itself or usage
of the home network.  The primary objective of such initiatives is to
gain an understanding of access networks to enact better policies.  They
do not perform passive monitoring of the home network.

\fp {\bf Qualitative design studies of home networks.}  There are
previous studies on understanding home networks and attempts to design
better
systems~\cite{grinter2005work,grinter2009ins,keith2011advancing,Chetty:2007:SHL,poole2008more,calvert2007moving}.
These previous studies are qualitative: they rely exclusively on human
subject interviews and analysis on human interactions to identify
problem areas and to suggest better designs.  We offer a {\em
  quantitative} complement to these studies. We analyze passive
measurement data that is automatically collected and reported, which are
in turn used to derive meaningful observations about home networks that
may not be obvious or even revealed through studies with human
subjects.  The automated nature of our data collection and monitoring
allows us to observe longitudinal behavior and usage patterns, and it may
in some cases result in more accurate data about human activity and
network usage, since many of the questions that previous studies have
asked in interviews could be more completely and accurately addressed by
measuring the network traffic itself.

Previous studies have also built tools that improve interactivity and
aid troubleshooting in a home network environment, for example
measuring and displaying bandwidth usage and throughput in a home
network~\cite{Chetty-2010,Chetty:2011:WMI}.  Our work analyzes data from
{\em existing} networks, rather than trying to deploy new tools and
observe how usage changes as a result of those tools.  We analyze a
wider variety of network features, including wired vs. wireless usage,
number of active devices, diurnal patterns, and availability.

\fp {\bf Behavioral studies of Internet usage in \developing{} countries.}
Chen \ea studied the effect of sporadic and slow connectivity on user
behavior and found a better Web interaction model for such
environments~\cite{Chen:2010:CWI}. Wyche \ea performed a qualitative
study of how Kenyan Internet users adapt their usage behavior where
Internet connectivity is a scarce resource in terms of availability,
cost, and quality~\cite{Wyche:2010:DIC}.  Smyth \ea performed a
qualitative study on sharing and consuming entertainment media on mobile
phones in urban India~\cite{Smyth:2010:TTW}.  The data that we
gathered in \developing{} countries could help
corroborate some of these studies.



%\fp{\bf Measuring home networks}~\cite{DiCioccio:2012,www-netalyzr,Kreibich2010,www-bismark,sundaresan2011,Dischinger:imc2007,Sanchez2013}\\
%\fp{\bf Understanding home networks for better design}~\cite{calvert2007moving,grinter2005work,grinter2009ins,keith2011advancing}\\
%\fp{\bf Help in interacting, troubleshooting}~\cite{poole2008more,grinter2005work,chetty2008getting,Chetty:2007:SHL}\\
%\fp{\bf Passive measurements}~\cite{DiCioccio2013}\\
%\fp{\bf Power management}~\cite{chetty2008getting,Chetty:2009:EGU}\\
%\fp{\bf Speed and bandwidth}\\
