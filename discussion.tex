\section{Discussion}\label{sec:discussion}

This study offers a glimpse into
the characteristics of a variety of home networks; the findings in
the paper suggest many avenues for future work.

\paragraph{Combining network measurement with qualitative studies about
  Internet use.}  Previous work in home networking has explored various
characteristics of home networks via detailed user studies and
interviews~\cite{Chetty-2010,Chetty:2011:WMI}. Our results can
complement these studies, which have been primarily qualitative to date.
Future work might entail doing a study that jointly performs user
studies in conjunction with network traffic monitoring, to determine
whether users' perceptions about their network use are consistent with
the reality (\eg, whether people spend more or less time online than
they claim).

\paragraph{Device fingerprinting for security alerts.}  Various Internet
service providers offer services that alert users about possible
infected devices in the home network; unfortunately, because ISPs
typically cannot map offending traffic to a particular MAC address, it
is difficult for them to attribute traffic to a particular device.
Future work could follow up on device fingerprinting using
traffic patterns to develop a system that provides more fine-grained
alerts to Internet service providers about the suspicious activities of
an individual device within a home.

\paragraph{Expanding the study of usage to more countries.}  
Our study of home network usage focused on the United States
alone.  Future work could expand this part of the study to determine how
usage patterns and other traffic characteristics (\eg, device usage,
popular domains) differ by country.  

%% We have much ongoing work on this project. We are currently expanding
%% our study in three ways: (1)~deploying in more countries (2)~increasing
%% recruitment for our passive data collection, and (3)~expanding the set
%% of measurements and analyses that run on the routers. In particular, we
%% are studying in greater detail the effects of the wireless network on
%% overall home network performance, and how interference and contention
%% affect the quality of the network. We are also searching for causes of
%% low access link utilization. Potential reasons, apart from lack of
%% demand, could be the nature of content delivery today (small web
%% objects, allied with longer latencies). 

%% This study can complement both traditional networking research, which
%% typically concerns itself with large-scale quantitative technical
%% measurements, and HCI research, which tends to favor small-scale yet
%% thorough case studies. Moreover, we believe our methodology and
%% deployment infrastructure itself can help fellow researchers collect
%% useful data and deduce meaningful conclusions and implications that will
%% eventually help design a better home network infrastructure and related
%% technologies fit for ubiquitous computing. We plan to open our platform
%% to the public so researchers around the world can measure and analyze
%% network data from a global collection of households.
