\section{Introduction}

Home broadband Internet access is ubiquitous and rapidly evolving. There
are now upwards of one billion broadband Internet users
worldwide~\cite{www-internet-stats}; much of that usage is shifting away
from conventional desktops and towards mobile devices, such as laptops,
smart phones, and tablets~\cite{www-itu}.  Despite their pervasiveness,
little is known about most home networks or how people use them.

Indeed, to date, it has been surprisingly difficult to study home networks on
a large scale, because network technologies like network
address translators (NATs) present only an opaque view of the home
network to the global Internet---specifically, without a monitoring
device {\em inside} the home, traffic coming from any device in a
home network appears to all be coming from a single device.  This coarse
granularity of visibility makes it impossible to observe the usage
patterns of individual devices inside the home or observe
characteristics about other parts of a home network (\eg, the home
wireless network).  Observing home networks on a large scale requires
developing---and deploying---an always-on monitoring device in the home
that can capture information about individual devices, with the consent of
the people that live in these homes.

To better understand home networks, we developed \name{}, a custom home router, and
deployed it in more than 100 home networks in 21 countries around the world for
more than one year.  \name{} sits between the user's ISP access link and the
rest of the home network and acts as a continuous monitoring device; as a
result, it can observe all traffic entering and leaving the home network, and
can attribute traffic flows to individual devices on the network.  This
device, if permitted, can also log other types of activity, such as the number of devices on
the network at any given time and the amount of traffic that any particular
device sends to a destination (\eg, Google).  It can also independently measure
the performance of both the home wireless network and the access link.
%To better understand home networks, we developed a custom home router and
%deployed it in more than 100 home networks in 21 countries around the world for
%more than one year.  The router sits between the user's ISP access link and the
%rest of the home network and acts as a continuous monitoring device; as a
%result, it can observe all traffic entering and leaving the home network, and it
%can also attribute traffic flows to individual devices on the network.  This
%device, if permitted, can also log other types of activity, such as the number of devices on
%the network at any given time and the amount of traffic that any particular
%device sends to a destination (\eg, Google).  It can also independently measure
%the performance of both the home wireless network and the access link.
We study three aspects of home network usage:
\begin{enumerate}
\itemsep=-1pt
\item {\em Availability.} How common are Internet connectivity outages in homes,
    and how reasonable is it to assume continuous connectivity?
\item {\em Infrastructure.} What networking technologies and devices do people use
  in home networks?
\item {\em Usage.} Is the capacity of a home network sufficient for the
  growing demands of users and applications?  How does usage differ across
  individual devices in the home?  What other patterns exist?
\end{enumerate}
\noindent
%Each of these characteristics has important implications for ubiquitous
%computing.
\noindent
Although our study instruments a relatively small number of home networks,
it nevertheless offers extensive visibility into aspects of homes that were
previously opaque to researchers.  Ultimately, regulators and Internet service
providers may be able to use some of the techniques that we describe to
perform similar studies on a larger scale.  We believe that researchers,
ISPs, policymakers, and users can use the home router as a measurement
device to better understand various trends in availability,
infrastructure, and usage.  Data about availability can help regulators
determine whether ISPs are delivering the service that they are
promising to users.  Data about infrastructure (\eg, the amount of
contention in the wireless spectrum) can help ISPs better understand and
debug user performance, and can provide evidence for regulators
to release more spectrum when it becomes appropriate to do so.
Information about usage can help ISPs better provision and plan, and it
can help device designers better understand how (and when) people use
various devices.

%shorten these three paragraphs?
The {\em availability} of home broadband access networks lets developers
of applications running over edge networks understand the connectivity
conditions of a particular environment.  Our study has yielded some
surprising findings. For example, although in the Western world we often
think of broadband connectivity as ``always on'', we found examples in
China and India where users only power up their home network gateway
when they intend to use it.  We found that trends of availability
held in general: only 10\% of home networks in the \developed{} world saw
connectivity interruptions of more than ten minutes more frequently than
once every 10 days, but about 50\% of home networks in \developing{}
countries experienced such connectivity interruptions once every 3 days.
Although it is difficult from our data set to ascertain the cause for
this intermittent connectivity (\ie, it could result from power outages,
poor connectivity, or behavioral patterns), it is clear that it is not
safe to assume that Internet connectivity is highly available in certain
parts of the world.

The {\em infrastructure} of home broadband access networks lets us understand
how users construct their home networks, and what types of devices they
comprise.  We explore various questions relating to infrastructure, such as the
number of devices on each home network, and how the number of devices on these
networks varies over time with usage.  We find that in today's home network, there
are more wireless devices than wired devices in
general and this difference is greater in the \developing{} countries.  
We also find households in more \developed{} countries tend to have more devices 
compared to \developing{} countries, and
those devices are more likely to remain continuously connected to the home router.
Connectivity shows a diurnal pattern across a week;
the number of devices on the network on
weekdays typically peaks during evening hours, but on weekends device usage is
more consistent throughout the day.  The median
number of devices on a 2.4~GHz network is about five, whereas on the 5 GHz band,
the median number of devices is two.
  

The {\em usage} of home broadband Internet access can shed light on the
applications and devices that people tend to use on their home network, and the
overall utilization of these networks. We analyze 
the periodicity of usage patterns in various home networks and observe the extent
to which users saturate their access link and find that most home networks are
lightly used, and do not saturate their downstream or upstream link most of the
time.  We analyze distributions of device usage and find that users normally
have a subset of devices that they prefer to use for consuming most network
data. %mostly one device hogs all the bandwidth
We observe the diversity of domains visited and find that about 38\% of the
total volume of traffic is from a single most popular domain, among 200
whitelisted domains.

The rest of the paper is organized as follows.
Section~\ref{sec:related} overviews related work.
Section~\ref{sec:data} describes the process of collecting the various
types of data that we use in our study.  Section~\ref{sec:availability}
presents results concerning the availability of broadband access in
various home networks around the world, including the duration of
outages.  Section~\ref{sec:infrastructure} describes the characteristics
of the infrastructure in various home networks, including the number of
devices that are in the network, whether the devices are connected
via wired or wireless, and to what extent devices occupy different
ranges of radio spectrum (\eg, 2.4 GHz and 5 GHz).
Section~\ref{sec:usage} profiles the usage patterns of users in
different home wireless broadband networks and explores how usage
patterns differ across devices, and how download speed affects usage
patterns.  Section~\ref{sec:discussion} describes ongoing work and
extensions to this study and discusses some broader implications of our
results.  Section~\ref{sec:conclusion} concludes.
