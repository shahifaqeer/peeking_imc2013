\section{Conclusion}\label{sec:conclusion}

Despite the proliferation of home networks, very little is known about
the properties of these networks, in terms of their availability,
infrastructure, or usage.  Although continual, longitudinal measurements
of these networks can provide insight into how people build, configure,
and use these networks, there has been little attempt to instrument
these networks to gather such data.  This paper represents the first
attempt to instrument a significant number of home networks to learn
about their properties. We presented the first large-scale, longitudinal
measurement study of home networks, based on data from 126 homes and 19
countries.  Our set of passive and active measurements allows us to
characterize properties of these networks such as network availability,
infrastructure, and usage.  We have publicly released all data sets that
do not involve personally identifying information, and we plan to
continue expanding the deployment into a broader, more diverse set of
environments.

Our study yields many interesting findings with respect to the
availability, infrastructure, and usage of home networks that could have
broader implications for ISPs, users, and policymakers.  With respect to
availability, we found that \developing{} countries experience far more frequent
connectivity interruptions, some of which are due to poor connectivity,
but others that are due to behavioral patterns (\eg, turning the router
on only during times when a user wants to access the Internet).  More
insights into how behavioral patterns differ across countries may help
both ISPs and application designers.  We also found that the 2.4~GHz
spectrum is significantly more crowded than the 5~GHz spectrum, both in
terms of number of devices and in terms of the number of visible access
points; more widespread statistics about the usage of wireless spectrum
(as will hopefully be possible as we continue to expand the \name{}
deployment) can ultimately help ISPs debug connectivity problems in home
networks and provide policymakers important data about spectrum usage.
Finally, we see that most of the traffic from homes is destined for only
a few domains, and that, on average, most of the traffic originates from
just a small handful of devices.  These usage statistics may ultimately
help ISPs with provisioning and peering decisions, and they also offer a
rare picture into how people use and interact with their home networks.
Although this study has offered a first glimpse into many aspects home
networks around the world, we expect that more lessons will come with
more experience and a broader deployment.


%% Our study yields many takeaways:
%% \begin{itemize}
%% \item  Certain parts of the world
%% may have significantly higher frequency of outages, which could be
%% affected by both behavioral patterns and power or network outages. 
%% %\item Wired devices are not as 
%% %popular as wireless devices in the \groupb{} regions.
%% %TODO we can give average numbers here...
%% \item Wired devices are not as popular as wireless devices in general.
%% \item Home networks in \developed{} countries generally have 
%% more devices that never disconnect.
%% \item The 2.4~GHz wireless spectrum
%% is significantly more crowded in the \developed{} countries.
%% \item From our passive data analysis, we saw that access links are significantly
%%     underutilized; in most homes, 90\% of the time the network used less than
%%     50\% of the available throughput.
%% \item The most popular domains by volume of traffic are not the
%% most popular domains by number of connections. 
%% \item Even in homes with multiple devices, there is a single
%% dominant device responsible for most network activity.
%% \end{itemize}
 
